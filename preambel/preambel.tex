\setcounter{secnumdepth}{3}		% 3 Ebenen bei der Kapitelnumerierung

%% Pakete -------------------------------------------------------------------------------
\usepackage{morewrites}
\usepackage[ngerman]{babel}		% neue deutsche Rechtschreibung
\usepackage{upgreek}
\usepackage{textcomp}
\usepackage{siunitx}
\usepackage{lmodern} 
\usepackage{blindtext}				% Mit \blindtext wird Dummytext eingefügt
\usepackage{mdwlist}				% kompaktere Listen mit itemize* und co
\usepackage{graphicx}				% Grafiken
\usepackage{epstopdf}               % epsBilder automatisch in pdfBilder konvertieren
\usepackage{booktabs,tabularx}		% longtable und tabularx Pakete (XSpalten)
\usepackage{tabularx}				% array
\usepackage[most]{tcolorbox}
\usepackage{enumitem}
\usepackage[normalem]{ulem}
\usepackage[T1]{fontenc}
\usepackage{lscape}
\usepackage{multirow}		% Mehrzeiliges in Tabellen
\usepackage{rotating}		% Rotieren von Text & co
\usepackage{color}			    	% Alles in Bunt und Farbe
\usepackage{colortbl}
\usepackage{multicol}		% Mehrzeilig
\usepackage{amsmath}		% mathematische Formeln
\usepackage{amsthm}		% theoremstyle
\usepackage{amssymb}		% mathematische Symbole
\usepackage{wrapfig}			% umflossene figures
\usepackage{float}				% für [h!] bzw. [H] positionierung
\usepackage{booktabs}	    % Für Excel2Latex Tabellen
\usepackage{eurosym}		% Eurosymbol verwenden
\usepackage{url}				% Weblinks einbinden
\usepackage{fancybox}		% Kästen und Boxen
\usepackage{caption}         % Erweiterte Einstellung für Beschriftungen
\usepackage[list=true]{subcaption}
\usepackage{array}             % erweiterte Optionen
\usepackage{pdfpages}		% Einbinden von pdf-Dateien möglich
\usepackage{setspace}		% lokales Ändern des Zeilenabstandes möglich
\usepackage{acronym}		% Abkürzungen Lang- und Kurzschreibweise
\usepackage{todonotes}		% Todo Kommentarte
\usepackage{listings}			% Quellcode

%% PDF Optionen (Verweise und co) -----------------------------------------------------------------
\usepackage{ifxetex}
\ifxetex
\usepackage[babel. german=quotes]{csquotes}
\usepackage{xltxtra}
\usepackage{etex}
\usepackage[dvipsnames]{pstricks}
\usepackage{auto-pst-pdf}
\usepackage{pdfpages}
\usepackage{lscape}
\usepackage{pdflscape}
\usepackage[pdfa, xetex, breaklinks=true]{hyperref}
\else
\usepackage[utf8]{inputenc}			% UTF-8 encoding
\usepackage[babel, german=quotes]{csquotes}
\usepackage[pdftex, raiselinks, pdfpagelabels, hypertexnames=false, pdfstartview={Fit}, breaklinks=true]{hyperref}

%% Angaben für die PDF-Datei
\hypersetup{
	pdfauthor = {Vorname Nachname},
	pdftitle =  {Master-Thesis},
	pdfsubject = {Titel},
	pdfkeywords = {Keyword1. Keyword2. ...},
	pdfcreator = {LaTeX with hyperref package}
}

%% Kopf- und Fußzeile -----------------------------------------------------------------------------
\usepackage[headsepline,automark,komastyle,nouppercase]{scrpage2}
\pagestyle{scrheadings}
\clearscrheadings % löschen der Stile
\clearscrplain
\ohead{\pagemark} % Neudefinition
\ihead{\headmark}
\cfoot{}
\automark[section]{chapter} % Kopfzeile: [rechts]{links}

%% Seitenränder -----------------------------------------------------------------------------------
\usepackage[a4paper,left=4cm,right=2.5cm,top=2cm,bottom=2cm,includeheadfoot]{geometry}

%% Absatzabstand -----------------------------------------------------------------------------------
\setlength{\parindent}{0pt}

%% Literaturverzeichnis ---------------------------------------------------------------------------
\usepackage[style=alphabetic,backend=bibtex]{biblatex}
\addbibresource{Literatur.bib} % Bibliographie laden

%% Verzeichnis für Listings
% Listings umbenennen
\renewcommand\lstlistingname{Algorithmus}
\renewcommand\lstlistlistingname{Algorithmenverzeichnis}
\def\lstlistingautorefname{Alg.} 


% Pfad für Bilder
\graphicspath{{assets/}}

%% Appendix -------------------
\usepackage[toc,page]{appendix}
